\documentclass[conference]{IEEEtran}
\IEEEoverridecommandlockouts
% The preceding line is only needed to identify funding in the first footnote. If that is unneeded, please comment it out.
\usepackage{cite}
\usepackage{amsmath,amssymb,amsfonts}
\usepackage{algorithmic}
\usepackage{graphicx}
\usepackage{textcomp}
\usepackage{xcolor}
\def\BibTeX{{\rm B\kern-.05em{\sc i\kern-.025em b}\kern-.08em
    T\kern-.1667em\lower.7ex\hbox{E}\kern-.125emX}}
\begin{document}

\title{Automatized Segmentation of Cracks in Solar Cells\\
}

\author{\IEEEauthorblockN{1\textsuperscript{st} Given Name Surname}
\IEEEauthorblockA{\textit{dept. name of organization (of Aff.)} \\
\textit{name of organization (of Aff.)}\\
City, Country \\
email address}
\and
\IEEEauthorblockN{2\textsuperscript{nd} Given Name Surname}
\IEEEauthorblockA{\textit{dept. name of organization (of Aff.)} \\
\textit{name of organization (of Aff.)}\\
City, Country \\
email address}
\and
\IEEEauthorblockN{3\textsuperscript{rd} Given Name Surname}
\IEEEauthorblockA{\textit{dept. name of organization (of Aff.)} \\
\textit{name of organization (of Aff.)}\\
City, Country \\
email address}
\and
\IEEEauthorblockN{4\textsuperscript{th} Given Name Surname}
\IEEEauthorblockA{\textit{dept. name of organization (of Aff.)} \\
\textit{name of organization (of Aff.)}\\
City, Country \\
email address}
\and
\IEEEauthorblockN{5\textsuperscript{th} Given Name Surname}
\IEEEauthorblockA{\textit{dept. name of organization (of Aff.)} \\
\textit{name of organization (of Aff.)}\\
City, Country \\
email address}
\and
\IEEEauthorblockN{6\textsuperscript{th} Given Name Surname}
\IEEEauthorblockA{\textit{dept. name of organization (of Aff.)} \\
\textit{name of organization (of Aff.)}\\
City, Country \\
email address}
}

\maketitle

\begin{abstract}

\end{abstract}

\begin{IEEEkeywords}
Crack detection,
\end{IEEEkeywords}

\section{Introduction}
\begin{itemize}
	\item Some basics and marketing
	\item  State-of-the-art methods
	\item Many algorithms have been proposed for crack detection in Si-PV, but no standard or reference algorithm has been established so far
	\item The computer science standard to publish the code (use for comparison) is often not followed  
	\item Aim of this work: establish a reference algorithm for scientific community which may be used as benchmark
\end{itemize}
 

\section{Materials \& Methods}
\begin{itemize}
	\item Samples 
	\subitem EL
	\subitem Polycrystalline because more difficult for automatized segmentation
	\subitem Number of images
	
	\item Preprocessing
	\subitem Fourier Filtering
	\subitem ROI of the solar cell
	\subitem Noise-reduction filters
\end{itemize}

\subsection{Crack Detection}
\begin{itemize}
	\item BP description (as one of the computer vision techniques used in industry)
	\item Vesselness
	\item  Adapted Vesselness
\end{itemize}

\section{Results and Discussion}
\begin{itemize}
	\item detection of ROI
	\item labeling in order to obtain performance result
	\item BP kernel size-threshold VS performance results
	\item Vesselness at different scales \& threshold
	\item Adapted Vesselness results at various degree of asymmetries
	\item Final results of AV with a denoising filter
\end{itemize}


\section{Conclusion}


\section{References}



\end{document}
